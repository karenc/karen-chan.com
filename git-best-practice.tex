\documentclass[landscape]{slides}

%\usepackage[paperheight=21cm,paperwidth=36cm]{geometry}
\usepackage{color}
\usepackage{fullpage}
\usepackage{pdfpages}
\usepackage{hyperref}

\begin{document}

\begin{slide}
\begin{center}
\textbf{\huge Git Best Practice:}
\end{center}
\begin{center}
\textbf{\huge Commit History}
\end{center}
\vspace{15mm}
\begin{center}
Karen Chan\\
\vspace{10mm}
December 2012
\end{center}
\end{slide}

\begin{slide}
\textbf{In this session we will talk about:}
\begin{itemize}
\item Why it is good to have a clean and linear history
\vspace{-10mm}
\item What is a good commit message
\vspace{-10mm}
\item What is a clean history
\vspace{-10mm}
\item How to read the git log graph
\vspace{-10mm}
\item How to avoid merge commits
\vspace{-10mm}
\item How to clean up your commits / rewrite history
\end{itemize}
\end{slide}

\begin{slide}
\textbf{Why is it good to have a clean and linear history}
\begin{itemize}
{\item So we can read and understand what has been changed just by reading git log,}
\vspace{-10mm}
{\item Makes it easier to debug which commit caused a bug,}
\vspace{-10mm}
{\item Makes it easier to rollback or cherry-pick changes}
\vspace{-10mm}
{\item With a good commit message we can remember why\\ something was changed}
\end{itemize}
\end{slide}

\begin{slide}
\textbf{What is a good commit message}\\
\begin{verbatim}
One sentence summary of what you did

More details about what you have done, for example why this
change exists, what you have deleted and why (which you can't
put as a comment in the code)
\end{verbatim}
It is easier to rely on just git commit messages instead of tickets, so these
commit messages are too brief:\\
\vspace{-18mm}
\begin{itemize}
\item{\tt 12345 Add tests}
\vspace{-10mm}
\item{\tt 23456 Typo}
\vspace{-10mm}
\item{\tt 34567 Fixed bug}
\end{itemize}
In fact if you include everything, then the ticket number becomes much less
important.
\end{slide}

\begin{slide}
\textbf{What is (not) a clean history}
\begin{enumerate}
{\item {\tt Add comment validation for MADA submissions}}
\vspace{-10mm}
{\item {\tt Remove debug statements that were in the last commit}}
\vspace{-10mm}
{\item {\tt Fix typo}}
\vspace{-10mm}
{\item {\tt One of the tests failed, fixed it}}
\vspace{-10mm}
{\item {\tt Forgot to add the test files}}
\end{enumerate}
Do you really care about commits 2 - 5?\\
Is it useful for you to see all the mistakes I have made?\\
\end{slide}

\begin{slide}
\textbf{But I want to commit regularly...}\\
\\
Don't worry, git gives you the ability to commit whenever you want, but also to
clean it up before you publish your changes.\\
\\
Git lets you rewrite history!
\end{slide}

\begin{slide}
\textbf{A linear history}\\
\\
Just having clean commits isn't really enough, having lots of merges also make
the history look crazy:\\
\\
\tt{git log --oneline --decorate --graph origin/production}
\end{slide}

\begin{slide}
\textbf{Once upon a time, when history was linear}\\
\\
\tt{* \textcolor{yellow}{33784f7} Fixing the build (same packet...}\\
\tt{* \textcolor{yellow}{624935a} Converting null amounts to zero when...}\\
\tt{* \textcolor{yellow}{7c42fd7} Changing deployment recipes to use git}\\
\tt{* \textcolor{yellow}{3592704} [Ed] added empty data folders}
\end{slide}

\begin{slide}
\textbf{A simple merge that we can still understand}\\
\\
\tt{*   \textcolor{yellow}{bbbc5ef} Merge branch 'master' of svn.groupm.local...}\\
\textcolor{blue}{\tt{|}}\textcolor{red}{\textbackslash}\\
\tt{\textcolor{blue}{|} * \textcolor{yellow}{60763cb} Story \#198: CRUD pages ...}\\
\tt{* \textcolor{red}{|} \textcolor{yellow}{afe2440} changing column to bigint}\\
\tt{* \textcolor{red}{|} \textcolor{yellow}{7ae049f} Adding new table for duns...}\\
\tt{* \textcolor{red}{|} \textcolor{yellow}{3c1047f} Renaming InvestigationService...}\\
\textcolor{red}{\tt{|/}}\\
\tt{* \textcolor{yellow}{6e04551} Finishing refactoring to add product type to...}\\
\end{slide}

\begin{slide}
\textbf{Getting complicated now...}\\
\\
\tt{*   a4f2cfc (dev1/production) Merge branch `staging' into...}\\
\tt{\textcolor{blue}{|}\textcolor{magenta}{\textbackslash}  }\\
\tt{\textcolor{blue}{|} *   fdc899b Merge branch `ticket74722' into staging}\\
\tt{\textcolor{blue}{|} \textcolor{cyan}{|}\textcolor{red}{\textbackslash}  }\\
\tt{\textcolor{blue}{|} \textcolor{cyan}{|} * c4f26c6 74722 Remove region filter on MADA expected...}\\
\tt{\textcolor{blue}{|} * \textcolor{red}{|}   cacdd8a Merge branch `ticket74328' into staging}\\
\tt{\textcolor{blue}{|} \textcolor{red}{|}\textcolor{yellow}{\textbackslash} \textcolor{red}{\textbackslash}  }\\
\tt{\textcolor{blue}{|} \textcolor{red}{|} \textcolor{yellow}{|}\textcolor{red}{/}  }\\
\tt{\textcolor{blue}{|} \textcolor{red}{|/}\textcolor{yellow}{|}   }\\
\tt{\textcolor{blue}{|} \textcolor{red}{|} * dba4d20 74328 Only do one duns investigation request...}\\
\tt{* \textcolor{red}{|} \textcolor{yellow}{|}   75d6ff0 Merge branch `staging' into production}\\
\end{slide}

\begin{slide}
\includepdf{git_log_graph.pdf}
\end{slide}

\begin{slide}
\textbf{Merge commits}\\
\\
The more layers we have the more merge commits we have.  Usually merge commits
are empty, and are not that useful to anyone.\\
\\
A few of them is fine, but when there are merge commits every 2 actual commits
it is useless noise and is distracting.
\end{slide}

\begin{slide}
\textbf{How to avoid merge commits when using git pull}\\
\\
\begin{itemize}
\item{Do not use {\tt --no-ff} when merging\\
({\tt [merge] ff = true} in .gitconfig)}
\item{Use {\tt rebase} if you have changes on your branch\\
({\tt [pull] rebase = true} in .gitconfig)}
\item{Use {\tt git push origin HEAD} to push only the current branch\\
({\tt [push] default = current} in .gitconfig)}
\end{itemize}
\end{slide}

\begin{slide}
\textbf{How to clean up your commits / rewrite history}\\
\begin{itemize}
\item{\tt{git commit --amend}}
\vspace{-10mm}
\item{\tt{git rebase}}
\vspace{-10mm}
\item{\tt{git rebase -i}}
\vspace{-10mm}
\item{\tt{git add -p}}
\end{itemize}
\end{slide}

\begin{slide}
\textbf{git commit {\tt--}amend}\\
\\
This lets you amend your last commit.\\
This is one of the most useful commands in git.\\
\\
For example, you just committed, then you realized you\\ forgot to add a file:\\
\\
{\tt git add missing\_file}\\
{\tt git commit --amend}\\
\\
There, you just added missing\_file to the last commit instead of doing a
commit like ``Forgot to add missing\_file''
\end{slide}

\begin{slide}
\textbf{git rebase (1)}\\
\\
Rebasing is really important to keep a linear history.\\
\\
For example, if I create a branch off {\tt qa}, then someone pushes one commit
to {\tt qa}, then I commit to my branch, the commit history will look something
like this:\\
\\
\tt{\textcolor{blue}{|} * \textcolor{yellow}{60763cb} My commit}\\
\tt{* \textcolor{red}{|} \textcolor{yellow}{3c1047f} Commit 2}\\
\textcolor{red}{\tt{|/}}\\
\tt{* \textcolor{yellow}{6e04551} Commit 1}\\
\end{slide}

\begin{slide}
\textbf{git rebase (2)}\\
\\
If you now merge into {\tt qa} instead of doing {\tt git rebase qa} first, the
{\tt qa} history will be like this:\\
\\
\tt{*    Merge branch `my-branch'}\\
\tt{\textcolor{blue}{|} \textcolor{red}{\textbackslash}}\\
\tt{\textcolor{blue}{|} * \textcolor{yellow}{60763cb} My commit}\\
\tt{* \textcolor{red}{|} \textcolor{yellow}{3c1047f} Commit 2}\\
\textcolor{red}{\tt{|/}}\\
\tt{* \textcolor{yellow}{6e04551} Commit 1}\\
\end{slide}

\begin{slide}
\textbf{git rebase (3)}\\
\\
If you do git rebase qa, the qa history will be linear because you commit will
go on top of the latest commit on qa:\\
\\
{\tt* \textcolor{yellow}{60763cb} My commit}\\
\textcolor{red}{\tt{|}}\\
{\tt* \textcolor{yellow}{3c1047f} Commit 2}\\
{\tt* \textcolor{yellow}{6e04551} Commit 1}\\
\\
You have avoided a not very useful ``Merge my-branch into qa'' commit.\\
\end{slide}

\begin{slide}
\textbf{git rebase (4)}\\
\\
What if I have a few branches that need to go back to qa, and I don't want to
merge one then rebase the next one etc?\\
\\
{\tt git cherry-pick qa..second-branch}\\
{\tt git cherry-pick qa..third-branch}\\
\\
Cherry-pick will take your changes and commit message and apply to the current
branch ({\tt qa}), it won't remember what the parent of the commit is.\\
\\
Which means qa is going to have a linear history.
\end{slide}

\begin{slide}
\textbf{git rebase {\tt-}i (1)}\\
\\
This can be used to edit commit messages, reorder commits, execute tests in
between commits etc.\\
Basically this is {\it THE} command to use for cleaning up history.\\
\\
It is usually run like this: {\tt git rebase -i origin/qa}\\
\\
Your text editor will show something like this:\\
\end{slide}

\begin{slide}
\textbf{git rebase {\tt-}i (2)}\\
\\
\textcolor{yellow}{pick} \textcolor{cyan}{a2b1945} \textcolor{magenta}{59860 Add turnover submitter and...}\\
\textcolor{yellow}{pick} \textcolor{cyan}{795730e} \textcolor{magenta}{59862 Create Turnover Declarations...}\\
\textcolor{yellow}{pick} \textcolor{cyan}{b52827a} \textcolor{magenta}{59862 Add tests for...}\\
\vspace{-15mm}
\begin{verbatim}
# Rebase e7af698..731c2cc onto e7af698
#          
# Commands:            
#  p, pick = use commit                               
#  r, reword = use commit, but edit the commit message
#  e, edit = use commit, but stop for amending          
#  s, squash = use commit, but meld into previous commit          
#  f, fixup = like "squash", but discard this commit's log message
#  x, exec = run command (the rest of the line) using shell
\end{verbatim}

The top commit will be the first to apply, all the way down to the last.
\end{slide}

\begin{slide}
\textbf{git add {\tt-}p}\\
\\
This is very useful for splitting up your change into different commits.\\
\\
{\tt git add -p README.txt}\\
\\
Allows you to edit the diff and lets you stage a subset of the changes you have
made.\\
\\
{\tt git diff --cached}\\
{\tt git commit} (no {\tt -a})\\
\\
This lets you commit what you have chosen and leave other changes for another
commit later.
\end{slide}

\begin{slide}
\textbf{A note on rewriting commit history}\\
\\
This should obviously only be done on your private local\\ development branch.\\
\\
Changing what is already on {\tt origin/qa} for example, is not\\ really ok.\\
\\
In fact, git usually gives you a warning when you do {\tt git push}:\\
\vspace{-15mm}
\begin{verbatim}
To ssh://risk@scm/home/risk/repositories/riskplatform.git
 ! [rejected]        HEAD -> qa (non-fast-forward)
error: failed to push some refs to `ssh://risk@scm/home/risk/...'
\end{verbatim}
\end{slide}

\begin{slide}
\textbf{Links}\\
\\
\\
These articles have more about git rebase, merge and commit message:
\vspace{-8mm}
\begin{itemize}
\item {\small \url{http://sandofsky.com/blog/git-workflow.html}}
\vspace{-10mm}
\item {\small \url{http://randyfay.com/content/rebase-workflow-git}}
\vspace{-10mm}
\item {\small \url{http://darwinweb.net/articles/the-case-for-git-rebase}}
\vspace{-10mm}
\item {\small \url{http://www.reviewboard.org/docs/codebase/dev/git/clean-commits/}}
\end{itemize}
\end{slide}

\end{document}
